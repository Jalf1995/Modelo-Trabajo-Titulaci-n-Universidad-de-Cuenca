
\begin{table}[!htbp] 
{\centering 
\setlength\extrarowheight{-1pt}
\setlength{\tabcolsep}{6pt} % default value is 6pt
\captionsetup{justification=centering}
\captionsetup{labelsep = colon}
  \caption{Descripción de los Datos de Sesgo y Curtosis} 
  \label{AnexoCuandro1} 
{
\begin{tabular}{@{\extracolsep{5pt}} cccccc} 
\\[-1.8ex]\hline 
\hline \\[-1.8ex] 
$\boldsymbol{i}$ & \textbf{Sesgo} & \textbf{Curtosis} & $\boldsymbol{i}$ & S\textbf{esgo} & \textbf{Curtosis} \\ 
\hline \\[-1.8ex] 
1 & -0.3412 & 5.4633 & 26 & -0.7550 & 6.7428 \\ 
2 & -0.1253 & 4.9871 & 27 & 0.0776 & 5.0380 \\ 
3 & -0.4308 & 6.7269 & 28 & -0.8126 & 6.4421 \\ 
4 & -0.4079 & 5.5543 & 29 & -0.5083 & 5.2737 \\ 
5 & -0.5339 & 5.6945 & 30 & -0.4947 & 8.2185 \\ 
6 & -0.6073 & 5.3631 & 31 & -0.4152 & 4.3238 \\ 
7 & -0.8138 & 7.0000 & 32 & -0.4909 & 4.4662 \\ 
8 & -0.3273 & 5.8194 & 33 & -0.6765 & 5.7369 \\ 
9 & -0.5648 & 5.6049 & 34 & -0.7603 & 6.7071 \\ 
10 & -0.5523 & 6.2844 & 35 & -0.6254 & 5.8223 \\ 
11 & -0.4022 & 5.1605 & 36 & -0.2124 & 4.8670 \\ 
12 & -0.6439 & 4.8704 & 37 & -0.8214 & 5.8699 \\ 
13 & -0.1540 & 5.1572 & 38 & -0.5129 & 5.0914 \\ 
14 & -0.5843 & 6.1288 & 39 & -0.3136 & 5.7513 \\ 
15 & -0.6938 & 6.8914 & 40 & -0.7751 & 6.0262 \\ 
16 & -0.8564 & 10.3045 & 41 & -0.6124 & 5.2650 \\ 
17 & -0.8005 & 8.1019 & 42 & -0.7862 & 7.1892 \\ 
18 & -0.6821 & 5.9738 & 43 & -0.5277 & 6.3044 \\ 
19 & -0.6829 & 5.9141 & 44 & -1.0622 & 8.0266 \\ 
20 & -0.5831 & 5.4781 & 45 & -0.7816 & 6.2043 \\ 
21 & -0.9119 & 6.9243 & 46 & -0.6463 & 6.0255 \\ 
22 & -0.6293 & 5.9926 & 47 & -0.6486 & 12.1402 \\ 
23 & -0.1016 & 8.6516 & 48 & -0.7810 & 5.4094 \\ 
24 & -1.0524 & 8.7598 & 49 & -0.7889 & 5.6384 \\ 
25 & -0.4985 & 5.2841 & 50 & 0.6467 & 3.2453 \\ 
\hline \\[-1.8ex] 
\end{tabular}\par} }
\begin{tablenotes}[flushleft,margin=1in]
\justifying\linespread{1}\small
\item\hspace*{-\fontdimen2\font}\\Comentario: En esta tabla se muestran los parámetros empíricos de Sesgo y Curtosis de los $49$ portafolios de industria y el Bono del Tesoro a un Mes, durante la muestra temporal de 600 meses desde enero de 1972 hasta enero de 2022. Estos resultados sirven como un complemento de la Tabla (\ref{TablaEst}) y además sirven para corroborar el comentario de la Figura (\ref{Distribuciones}).
\end{tablenotes} 
\end{table} 

% \captionsetup{margin=.3cm}



