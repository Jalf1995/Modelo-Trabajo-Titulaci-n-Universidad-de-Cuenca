%Esta es la portada del trabajo de titulación basada en la: GUÍA GENERAL PARA LA PRESENTACIÓN DE TRABAJOS DE TITULACIÓN https://www2.ucuenca.edu.ec/images/biblioteca/Gu%C3%ADa%20-%20Trabajo%20de%20titulación.pdf

% Tener en cuenta que este código en LateX, es solo una aproximación al formato original
\maketitle

\newpage

\hphantom{Hola!}% Espacio fantasma 

\vspace{-0.05\baselineskip}

\begin{center}
    \setstretch{0}
    \centering\includegraphics[width=8.7cm,height=1.69cm]{Imagenes/logo-UCuenca-GRANDE-JPG.jpg}
\end{center}

\vspace{.7\baselineskip} %Se utiliza para generar un espacio entre Logo y la línea 1. Notar que en la guía este espacio no se especifica.

\begin{center} %En este primer entorno de centrado contenemos las lineas "Universidad de Cuenca", "Facultad XX", "Carrera XX", "Título del trabajo de titulación". Cada línea se maneja como un párrafo independiente, cada línea tiene un interlineado de 2 con la línea anterior y la línea que sigue.
    \begin{espaciodoble}
        {\fontsize{18}{18}\selectfont \textbf{Universidad de Cuenca}} %línea 1
        
        {\fontsize{14}{14}\selectfont Facultad de XX} %línea 2
        
        {\fontsize{14}{14}\selectfont Carrera de XX} %línea 3
    \end{espaciodoble}
    
    \vspace{.5\baselineskip}%Se utliza para manterner interlineado de 2 entre la 3ra y 4ta línea.
    
    \begin{espaciosimple}
        {\fontsize{12}{14}\selectfont \textbf{Escriba aquí el título de su trabajo de titulación Escriba aquí el título de su trabajo de titulación Escriba aquí el título de su trabajo de titulación }} %línea 4
        
        %La guía establece que el título del trabajo de titulación requiere interlineado simple, a lo cuál yo interpreto como un interlineado de 1. 
    \end{espaciosimple} 
\end{center}

\vspace{1.2\baselineskip}%Se utiliza para generar espacio entre la 4ta y 5ta línea. Notar que en la guía este espacio no se especifica.

\begin{espaciosimple}%Comienza entorno para manejar las líneas que no están centradas, en este entorno el interlineado base-standar es simple, o sea, interlineado de 1.
    \begin{flushright}
        \begin{minipage}{5.8cm}
            {\fontsize{11}{11}\selectfont Trabajo de titulación previo a la obtención del título de Trabajo de titulación previo a la obtención del título de} %línea 5
        \end{minipage}
    \end{flushright}
    
    \vspace{.5\baselineskip}%Se utiliza para generar espacio entre la 5ta y 6ta línea. Notar que en la guía este espacio no se especifica.
    
    \hphantom{Hola!}%Se utiliza para generar espacio entre la 5ta y 6ta línea. Notar que en la guía este espacio no se especifica.

    \hphantom{Hola!}%Se utiliza para generar espacio entre la 5ta y 6ta línea. Notar que en la guía este espacio no se especifica.

    \begin{flushleft}
        {\fontsize{11}{13}\selectfont \textbf{Autores:}} %línea 6
       
        {\fontsize{11}{13}\selectfont {Nombres y Apellidos completos del Autor A}} %línea 7
        
        {\fontsize{11}{13}\selectfont {Nombres y Apellidos completos del Autor B}} %línea 8

        \hphantom{Hola!}%Se utiliza para generar espacio entre  Autores y Director. Notar que en la guía este espacio no se especifíca.

        {\fontsize{11}{13}\selectfont \textbf{Director:}} %línea 9
       
        {\fontsize{11}{13}\selectfont {Nombre del tutor}} %línea 10
        
        \item{ORCID:}\href{https://orcid.org}{\;{\includegraphics[width=.5cm,height=.5cm]{Imagenes/orcid_16x16.png}}\hspace{1mm}0000-0000-0000-0000} %línea 11
    \end{flushleft}

    \hphantom{Hola!}

    \hphantom{Hola!}

    \hphantom{Hola!}

    \hphantom{Hola!}

    \hphantom{Hola!}

    \begin{center}
        {\fontsize{11}{13}\selectfont \textbf{Cuenca-Ecuador}} %línea 12

        {\fontsize{11}{13}\selectfont 2011-11-11} %línea 13
    \end{center}

\end{espaciosimple}

