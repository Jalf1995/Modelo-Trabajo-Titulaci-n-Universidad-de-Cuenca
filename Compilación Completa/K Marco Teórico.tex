\section{Marco Teórico (Ejemplo)}

\subsection{La Aversión al Riesgo}

La aversión al riesgo se define como la falta de disposición de un individuo a participar en una lotería de riesgo puro, es decir, una lotería cuyo pago esperado es nulo $E[\varepsilon_{i}]=0$. El individuo rechaza este tipo de lotería, ya que hacerlo le proporciona mayor utilidad que aceptar participar en ella. Matemáticamente esta afirmación sólo puede ser cierta, si y sólo si, la función de utilidad $U(\cdot)$ es cóncava. Para demostrarlo imaginemos primero una lotería de riesgo puro con una estructura simple de dos pagos, tal que, $E[\varepsilon_{i}]=0=p_1(\varepsilon_1)+(1-p_1)(\varepsilon_2)=0$ y definamos $U(\cdot)$ como una función de la riqueza inicial $U(\Bar{\omega}_{0})\;|\;\Bar{\omega}_{0})\in{\mathbb{R}_+}$. Si la decisión del individuo es rechazar la alternativa riesgosa entonces,
\begin{equation}
\begin{aligned}
U(\Bar{\omega}_{0})>E[U(\Bar{\omega}_{0}+\varepsilon_{i})]=p_1U(\Bar{\omega}_{0}+\varepsilon_1)+(1-p_1)U(\Bar{\omega}_{0}+\varepsilon_2)
\end{aligned}
\end{equation}

El lado izquierdo de esta última expresión puede escribirse en términos de $U(\Bar{\omega}_{0}+E[\varepsilon_{i}])$, por lo tanto,
\begin{equation}
\begin{aligned}
U[\Bar{\omega}_{0}+p_1(\varepsilon_{1})+(1-p_{1})(\varepsilon_{2})]>E[U(\Bar{\omega}_{0}+\varepsilon_{i})]=p_1U(\Bar{\omega}_{0}+\varepsilon_1)+(1-p_1)U(\Bar{\omega}_{0}+\varepsilon_2)
\end{aligned}
\end{equation}

Esta expresión muestra que si introducimos los valores de la estructura de pago de la lotería, el valor de la utilidad esperada resultante de rechazar la lotería sería mayor que la utilidad esperada de aceptar la lotería, si y sólo si, $U(\Bar{\omega}_{0}))$ es una función estrictamente cóncava\footnote{Una función $f$ es estrictamente cóncava, si y sólo si,
\begin{equation*}
\begin{aligned}
\defconcavidad
\end{aligned}
\end{equation*}}. 

\subsubsection{Medidas de Aversión al Riesgo} 

Aunque todos los individuos aversos al riesgo rechazan alternativas de riesgo puro, precisamente porque la función de utilidad es cóncava, no sería  pertinente suponer que nunca estarían dispuestos a aceptar alguna alternativa  riesgosa. Para determinar las alternativas riesgosas que un individuo averso al riesgo si estaría dispuesto a aceptar, necesitamos entender su medida de aversión al riesgo, un concepto que se volverá indispensable para determinar las decisiones de portafolio más adelante \citep{10.1093/acprof:oso/9780190241148.001.0001}.
 



